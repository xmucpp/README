% !Mode:: "TeX:UTF-8"
\documentclass{beamer}

\mode<presentation>
{
\usetheme{Warsaw}

\setbeamercovered{transparent}
}

\usepackage[UTF8]{ctex} % 支持中文

% 分页Slide不编号
\setbeamertemplate{frametitle continuation}{}

% 插入多列
\usepackage{multicol}

% 插入链接
\usepackage{hyperref}
\hypersetup{urlcolor=blue}

% 插入图片
\usepackage{graphicx}

% 插入代码
\usepackage{xcolor}
\definecolor{mygray}{RGB}{245,245,245}

\usepackage{listings}
\lstset{language=Python}
\lstset{escapeinside=``}
\lstset{numbers=left}
\lstset{breaklines}
\lstset{backgroundcolor=\color{mygray}}

% 书签乱码解决方案
% 在文件D:/CTEX/MiKTeX/tex/latex/beamer/base/beamer.cls中找到:
% \PassOptionsToPackage{bookmarks=true,%
%   bookmarksopen=true,%
%   pdfborder={0 0 0},%
%   pdfhighlight={/N},%
%   unicode=true,%       <-------------- 加入此行
%   linkbordercolor={.5 .5 .5}}{hyperref}

% If you have a file called "university-logo-filename.xxx", where xxx
% is a graphic format that can be processed by latex or pdflatex,
% resp., then you can add a logo as follows:

% \pgfdeclareimage[height=0.5cm]{university-logo}{university-logo-filename}
% \logo{\pgfuseimage{university-logo}}



% Delete this, if you do not want the table of contents to pop up at
% the beginning of each subsection:

% If you wish to uncover everything in a step-wise fashion, uncomment
% the following command:

%\beamerdefaultoverlayspecification{<+->}

\begin{document}

\title{China's Prices Project at Xiamen University(CPP@XMU)}
\subtitle{爬虫技术简介}

% - Use the \inst{?} command only if the authors have different
%   affiliation.
%\author{F.~Author\inst{1} \and S.~Another\inst{2}}
\author{岳忠信}

% - Use the \inst command only if there are several affiliations.
% - Keep it simple, no one is interested in your street address.
\institute[Universities of]
{
厦门大学\quad 经济学院}

\renewcommand{\today}{\number\year 年 \number\month 月 \number\day 日}
\date{\today}

% This is only inserted into the PDF information catalog. Can be left
% out.
\subject{Presentations}

% title page
\begin{frame}
\titlepage
\end{frame}

% content
\begin{frame}
\frametitle{目录}
%\begin{multicols}{2}
\tableofcontents
%\end{multicols}
% You might wish to add the option [pausesections]
\end{frame}


\section{爬虫主要流程}
\subsection{请求}
\begin{frame}
\frametitle{请求}
\begin{itemize}
  \item 判断网页是静态还是动态:查看源代码
  \begin{itemize}
  \item 在HTML里找到数据:静态网页
  \itme 找不到数据:动态网页
  \end{itemize}
  \item 寻找入口(URL)
  \begin{itemize}
    \item 静态网页:直接请求网页URL
    \item 动态网页:通过审查元素寻找数据API接口(XHR或者JS)
  \end{itemize}
  \item 判断请求方式:查看网页请求类型
  \begin{itemize}
    \item Get请求:无数据上传,通常数据通过构造URL上传
    \item Post请求:需要上传请求数据
  \end{itemize}
  \item 工具:requests\footnotemark
  \footnotetext{\url{http://docs.python-requests.org/en/master/}}
\end{itemize}
\end{frame}

\subsection{解析}
\begin{frame}
\frametitle{解析}
\begin{itemize}
  \item 静态HTML:bs4\footnotemark
  \footnotetext{\url{https://www.crummy.com/software/BeautifulSoup/}}
  \item 动态:
  \begin{itemize}
    \item 标准JSON格式:json
    \item 字符串格式:re(正则表达式)等
  \end{itemize}
\end{itemize}
\end{frame}

\subsection{储存}
\begin{frame}
\frametitle{储存}
\begin{itemize}
   \item CSV格式:csv
   \item JSON格式:json
   \item SQL:python-sql
   \item MongoDB:pymongo
   \item \ldots
\end{itemize}
\end{frame}

\section{常见问题}
\subsection{寻找入口}
\begin{frame}
\frametitle{动态网页寻找入口}
\begin{itemize}
  \item 审查元素-Network:抓包
  \item 在XHR里面找JSON格式的数据
  \item 如果XHR没有,在JS里面找
\end{itemize}
\end{frame}

\subsection{反爬}
\begin{frame}[allowframebreaks]
\frametitle{反爬及应对策略}
\begin{itemize}
  \item 症状:
  \begin{itemize}
    \item 4XX:无法访问,通常是无法请求
    \item 3XX:重定向,通常是需要输入验证码或者需要登陆
  \end{itemize}
  \item 可能原因:
  \begin{itemize}
    \item 检查请求头
    \item 控制访问速率
    \item 需要用户登陆
  \end{itemize}
  \framebreak
  \item 初级解决方案:
  \begin{itemize}
    \item 代理UA
    \item 控制速率
    \item 换IP
    \item 识别验证码
    \item 模拟登录
  \end{itemize}
\end{itemize}
\end{frame}

\subsection{动态数据解析}
\begin{frame}
\begin{itemize}
  \item 主要问题:数据格式混乱
  \item 解决方案:正则表达式+技巧
\end{itemize}
\end{frame}

\subsection{翻页策略}
\begin{frame}
\begin{itemize}
  \item 构造URL
  \item 匹配''下一页''
  \item 判断是否重复爬取
\end{itemize}
\end{frame}

\subsection{处理编码}
\begin{frame}
\begin{itemize}
  \item 解决方案:
  \begin{itemize}
    \item 基本原则:输入程序解码成unicode,输出程序编码成其他编码
    \item 技巧:反复检查编码 
  \end{itemize}
\end{itemize}
\end{frame}

% \section{常见问题}
% 
% \subsection{静态爬虫}
% \begin{frame}
% \frametitle{步骤}
% \begin{enumerate}
%   \item 判断出它是静态内容
%   \item 利用requests库请求html.
%   \begin{itemize}
%   	\item 反爬
%   	\item 具体症状:返回的html错误或者出现403等报错数字
%   	\item 原因:一般,浏览器在向服务器发送请求的时候,会有一个请求头——User-Agent,它用来标识浏览器的类型.当我们使用requests来发送请求的时候,默认的User-Agent是python-requests/2.**。
%     \item 反爬策略:修改User-Agent,如修改为
%     \begin{lstlisting}
%     'Mozilla/5.0 (Macintosh; Intel Mac OS X 10_11_2) AppleWebKit/537.36 (KHTML, like Gecko) Chrome/47.0.2526.80 Safari/537.36'
%     \end{lstlisting}
%   \end{itemize}
%   \item 请求下正常的html之后,用BeautifulSoup进行解析。
% \end{enumerate}
% \end{frame}
% 
% \subsection{动态爬虫}
% \begin{frame}
% \frametitle{概述}
% 难点:
% \begin{itemize}
%   \item 如何获取url
%   \item 如何请求html(requests的使用)
%   \item 如何处理返回的html
% \end{itemize}
% \end{frame}
% 
% \begin{frame}
% \frametitle{步骤}
% \begin{enumerate}
%   \item 审查元素(建议使用谷歌浏览器)\\注意:刷新,浏览整个页面\\观察内容:一般在xhr或者js中,获取正确的url
%   \item 获取内容\\使用库:requests\\具体方法:
%   \begin{itemize}
%   	\item 观察headers,决定使用get或post以及对应的请求头\\(一般来讲,其中我们需要关注的是Cookie / Host / Origin / Referer / User-Agent / X-Requested-With等参数。)
%   	\item 好消息是,这样的请求往往得到的内容是json格式的,所以我们非但不会加重爬虫的任务,反而可能会省去解析HTML的功夫。
%   	\item 运用json库将其封装成字典
%   \end{itemize}
% \end{enumerate}
% 接下来我们就可以利用对python的list和dict的操作来获取我们想要的信息
% \end{frame}
% 
% \subsection{常见问题}
% \begin{frame}
% \frametitle{常见问题}
% \begin{enumerate}
%   \item 获取html时出现错误\\解决方案:观察headers\\判断是否是反爬(print出来状态码)解决方法:修改请求头
%   \item 解析html的坑:\\
%   \begin{itemize}
%   	\item 返回的不是正常的json格式\\方法:import re 用正则表达式洗字符串
%   	\item 编码错误\\查看编码:观察headers或print出来\\方法:转化成unicode.
%   	\end{itemize}
% \end{enumerate}
% 解决了这些问题,我们就应该能顺利地写出一篇单面爬虫了,那么如何翻页呢?
% \end{frame}
% 
% \begin{frame}
% \frametitle{翻页策略}
% \begin{enumerate}
%   \item 观察url,修改其中的参数来重新运行观察有什么效果
%   \item 找到最大面数,然后利用字符串格式化来修改url\\用迭代的方法写成循环。
%   \item 观察运行效果
% \end{enumerate}
% \end{frame}
% 
% \begin{frame}
% \frametitle{常见错误}
% \begin{enumerate}
%   \item 返回的是空list现在猜测原因是正则表达式的问题)
%   \item 编码错误(解决思路:从一开始请求时就将其转化为unicode)
%   \item ValueError
% \end{enumerate}
% \end{frame}
% 
% 


\begin{frame}
\begin{center}
  \includegraphics[width=7cm,height=7cm]{CPP.jpg}
\end{center}
\end{frame}

\end{document}
